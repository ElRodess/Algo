\documentclass [a4] {article}

\usepackage [utf8] {inputenc}
\usepackage [french] {babel}
\usepackage {amsfonts, amsmath, amssymb,amsthm,array,multirow,makecell}
\usepackage{hyperref}
\setcellgapes{1pt}
\makegapedcells
\usepackage[table]{xcolor}
\newcolumntype{R}[1]{>{\raggedleft\arraybackslash }b{#1}}
\newcolumntype{L}[1]{>{\raggedright\arraybackslash }b{#1}}
\newcolumntype{C}[1]{>{\centering\arraybackslash }b{#1}}

\author {D'Amonville Ronan Desquaires Joseph}
\title {Rapport de projet d'Algorithme}


\begin {document}
\maketitle

%Rappel du sujet%
\textsf Le projet a pour but de programmer un algorithme permettant aux skieurs de se diriger dans la station de ski de l'Alpe d'Huez.\\

%Langage Choisi%
\newpage
\section {Langage choisi}

\paragraph { Nous avons choisi le langage C car c'est le seul que nous maîtrisons pour l'instant, on aurait put s'orienter vers le langage Java avec plus d'expérience.} 

\section{Structure de données}

\paragraph {On a représenté le graphe sous la forme d'une matrice de la forme suivante :}
\begin{tabular}{|c|c|c|}
\hline  \cellcolor{black} & Sommet1 & Sommet2 \\
\hline Sommet1 & null & arête \\
\hline Sommet2 & null & null \\
\hline 
\end{tabular}
\paragraph{On remarque donc que si un Sommet peut allez vers un autre, on note le nom de la piste qu'il doit emprunter, sinon on marque null.}
\paragraph{Du coup a dût créer une structure piste qui est représenté ainsi :}
\begin{tabular}{|c|c|}
\hline  Une chaine de caractère & Son nom \\
\hline Un float & Le temps qu'il faut pour la parcourir (son poids) \\
\hline
\end{tabular}
\paragraph{Cela nous a permis de préciser le temps mais aussi quelles pistes emprunter pour allez à sa destination}
\section{principales étapes}
\paragraph{La première étape était de choisir quelle type de structure de données nous allions utiliser.Celle ci est décrite ci-dessus.}
\paragraph{La deuxième étape fût la plus longue : elle consistait à répertorier les différents sommets ainsi que les pistes dans la matrice, après quoi il fallait créer une base de donnée de pistes.}
\paragraph{La troisième étape a été de convertir ces 2 bases de données dans des fichiers textes, pour cela on a utiliser un programme fait maison afin de transformer notre matrice de nom de pistes en matrice de float. dans un fichier texte.}
\paragraph{Enfin il a fallut l'algorithme de Dijtra en C ainsi que l'adapter à notre structure de données}                                                                                                                                                                                                                                                                                                                                                                                                                                              
\newpage
\section{Fonctionnement de l'algorithme}
\paragraph{L'algorithme est expliqué en détails dans les commentaires du code, sinon il s'agit juste de l'algorithme de dijtra.}
\section{Difficultés rencontrées }
\paragraph{Les principales difficultées ont été la traduction du plan en structure de données ainsi que l'implémentation de Dijtra avec une matrice.}

\section{annexe}
\paragraph{Vous pouvez retrouver le dossier de notre projet sur https://github.com/ElRodess/Algo}                                                                                                                                                                                                                                                                                                                                                                                                                                                                                                                                                                                                                                                             

\end {document}


